This page describes the definitions of the most commonly used terminologies in Cyber R\-T.

\subsection*{Component}

In an autonomous driving system, modules(like perception, localization, control systems...) exist in the form of components under Cyber R\-T. Each component communicates with the others through Cyber channels. The component concept not only decouples modules but also provides the flexibility for modules to be divided into components based individual module design.

\subsection*{Channel}

Channels are used to manage data communication in Cyber R\-T. Users can publish/subscribe to the same channel to achieve p2p communication.

\subsection*{Task}

Task is the abstract description of an asynchronous computation task in Cyber R\-T.

\subsection*{Node}

Node is the fundamental building block of Cyber R\-T; every module contains and communicates through the node. A module can have different types of communication by defining read/write and/or service/client in a node.

\subsection*{Reader/\-Writer}

Message read/write class from/to channel. Reader/\-Writer are normally created within a node as the major message transfer interface in Cyber R\-T.

\subsection*{Service/\-Client}

Besides Reader/writer, Cyber R\-T also provides service/client pattern for module communication. It supports two-\/way communication between nodes. A client node will receive a response when a request is made to a service.

\subsection*{Parameter}

Parameter service provides a global parameter access interface in Cyber R\-T. It's built based on the service/client pattern.

\subsection*{Service discovery}

As a decentralized design framework, Cyber R\-T does not have a master/central node for service registration. All nodes are treated equally and can find other service nodes through {\ttfamily service discovery}. {\ttfamily U\-D\-P} is used in Service discovery.

\subsection*{C\-Routine}

Referred to as Coroutine concept, Cyber R\-T implemented C\-Routine to optimize thread usage and system reource allocation.

\subsection*{Scheduler}

To better support autonomous driving scenarios, Cyber R\-T provides different kinds of resource scheduling algorithms for developers to choose from.

\subsection*{Message}

Message is the data unit used in Cyber R\-T for data transfer between modules.

\subsection*{Dag file}

Dag file is the config file of module topology. You can define components used and upstream/downstream channels in the dag file.

\subsection*{Launch files}

The Launch file provides a easy way to start modules. By defining one or multiple dag files in the launch file, you can start multiple modules at the same time.

\subsection*{Record file}

The Record file is used to record messages sent/received to/from channels in Cyber R\-T. Reply record files can help reproduce the behavior of previous operations of Cyber R\-T. 